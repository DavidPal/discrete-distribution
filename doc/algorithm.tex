% LaTeX document describing the sampling algorithm.
%
% You will a recent LaTeX distribution installed to generate a PDF file.
% Run the following commands:
%
%   pdflatex algorithm.tex
%   bibtex algorithm.aux
%   pdflatex algorithm.tex
%   pdflatex algorithm.tex
%

\documentclass{article}

\usepackage{amsmath}

\usepackage{hyperref}
\hypersetup{colorlinks=true}

\title{Algorithm for Sampling from Discrete Distributions}
\author{D\'avid P\'al}

\begin{document}

\maketitle

\begin{abstract}
We describe an algorithm for generating random samples from any discrete
(categorical) distribution, given as an input, in $O(1)$ time per sample.  If
the distribution has $N$ categories, the algorithm needs $O(N)$ memory and
$O(N)$ pre-processing time.
\end{abstract}

\section{Introduction}
\label{section:introduction}

Generating random numbers from various probability distributions is an
important task in many scientific and industrial applications. In this short
note, we describe an algorithm for generating random numbers from a discrete
distribution, also called \emph{categorical
distribution}~\cite{wikipedia:categorical_distribution}.

A discrete distribution is specified by $N$ non-negative real numbers
$p_1, p_2, \dots, p_N$ that satisfy
$$
p_1 + p_2 + \dots + p_N = 1 \; .
$$
A sample from the distribution is a number from the set $\{1,2,\dots,N\}$.
Number $i \in \{1,2,\dots,N\}$ is generated with probability $p_i$.  The number
$N$ is called the number of categories.

In practical implementations, the algorithm is given as an input
a list of arbitrary non-negative numbers $w_1, w_2, \dots, w_N$
with positive sum and the goal is to generate samples from the discrete
distribution specified by $p_1, p_2, \dots, p_N$ where
$$
p_i = \frac{w_i}{\sum_{i=1}^N w_i} \; .
$$
Transforming $w_1, w_2, \dots, w_N$ to $p_1, p_2, \dots, p_N$ is easily
achieved in $O(N)$ time as part of the pre-processing phase.  In the rest of
the paper, we assume this has been done.

The problem of generating random numbers from a distribution is to design an
algorithm that receives numbers $p_1, p_2, \dots, p_N$ as an input,
pre-processes them, and is able to generate any number of independent samples
from the distribution. We make the standard assumption that the algorithm has
access to a random number generator that generates independent random real
numbers uniformly from the unit interval $[0,1]$ in $O(1)$ time per sample.

Naive, but fairly common, algorithm requires $O(N)$ pre-processing time, $O(N)$
memory, and can generate sample in $O(\log N)$ time.\footnote{All time and
memory complexities are in the worst-case sense.} As of year 2015, the naive
algorithm is used in popular implementations of the C++ standard
library~\cite{gcc-libstdc++,clang-libc++}.  We recap the algorithm in
Section~\ref{section:naive-algorithm}.

In Section~\ref{section:fast-algorithm}, we describe a faster algorithm that
uses $O(N)$ memory, has $O(N)$ pre-processing time, but requires only $O(1)$
time to generate a sample.  The space and time complexities are clearly
optimal.

\section{Naive Algorithm}
\label{section:naive-algorithm}

In the pre-processing phase, the naive algorithm computes prefix sums $s_0,
s_1, s_2, \dots, s_N$ where $s_0 = 0$ and for $i=1,2,\dots,N$
$$
s_i = p_1 + p_2 + \dots + p_i \; .
$$
The prefix sums can be computed in $O(N)$ time and use $O(N)$ memory.

To generate a sample the algorithm makes a single call to the random number
generator. Let $X$ be the number it obtains. Using
binary search, the algorithm finds an index $i \in \{1,2,\dots,N\}$ such that
$$
s_{i-1} \le X \le s_i \; .
$$
It is not hard to see that index $i$ is chosen with probability $s_i - s_{i-1}
= p_i$.

\section{Fast Algorithm}
\label{section:fast-algorithm}

The difference between the naive and the fast algorithm starts with the
pre-processing phase.  Given $p_1, p_2, \dots, p_N$, think of each $p_i$ as a
line segment of length $p_i$ and color $i$.  The algorithm cuts the line
segments into $2N$ smaller line segments. During cutting each point of each
line segment keeps its original color. Let $q_1, q_2, \dots, q_{2N}$ be the
lengths of the resulting line segments. Their length will satisfy
\begin{equation}
\label{equation:two-segments}
q_{2i - 1} + q_{2i} = \frac{1}{N}
\end{equation}
for $i=1,2,\dots,N$.  In other words, if we connect segments $q_{2i}$ and
$q_{2i-1}$ we get segment of length exactly $1/N$.

It is a non-trivial fact that any $p_1, p_2, \dots, p_N$ can be cut into $q_1,
q_2, \dots, q_{2N}$ satisfying equation \eqref{equation:two-segments}.
Furthermore, the cutting can be done in $O(N)$ time and $O(N)$ memory. We show
how to do the cutting in Section~\ref{subsection:cutting} below.

Once the cutting is done, the algorithm can generate a sample in $O(1)$ time:
First, it makes a single call to the random number generator generator. Let $X$
be the number it obtains. The algorithm computes an index $i \in \{1,2,\dots,N\}$
such that
$$
\frac{i-1}{N} \le X \le \frac{i}{N} \; .
$$
This can be done in constant time using formula $i = \lceil N \cdot X \rceil$
if $X$ is positive, and if $X = 0$, we set $i=1$.  Note that index $i$ is
uniformly distributed over $\{1,2,\dots,N\}$.  Number $X$ lies in the interval
$[\frac{i-1}{N}, \frac{i}{N}]$ of length $1/N$. We divide this interval into
two disjoint sub-intervals:
$$
\left[\frac{i-1}{N}, \frac{i}{N} \right]
=
\left[\frac{i-1}{N}, \frac{i-1}{N} + q_{2i-1} \right)
\cup
\left[\frac{i-1}{N} + q_{2i-1}, \frac{i}{N} \right] \; .
$$
The length of the first subinterval is $q_{2i-1}$ and the length of second
subinterval is $1/N - q_{2i - 1} = q_{2i}$. If $X$ lies in the first
subinterval, algorithm outputs the color of line segment $q_{2i-1}$. Otherwise,
$X$ lies in the second subinterval and the algorithm outputs the color of
line segment $q_{2i}$.

\subsection{How to Cut the Line Segments?}
\label{subsection:cutting}

We split the line segments $p_1, p_2, \dots, p_N$ into two types: short and
long. Short ones are those that have length smaller than $1/N$. Long ones have
length $1/N$ or bigger. Notice that there must be at least one long line
segment; since if all $N$ line segments were short, their total length
would be less than $1$.


The algorithm then consists of $N$ rounds. In each round $i=1,2,\dots,N$, it
constructs $q_{2i-1}$ and $q_{2i}$ satisfying \eqref{equation:two-segments}.
The construction of the pair $q_{2i-1}, q_{2i}$ will be done in $O(1)$ time.

In any round $i$, the algorithm takes an arbitrary short line segment and an
arbitrary long line segment. Let $s$ be the length of the short line segment
and let $\ell$ be the length of long line segment. We remove $s$ and $\ell$
from their respective piles. The line segment $q_{2i-1}$ will be the short
segment, i.e., $q_{2i-1} = s$ and the color of $q_{2i-1}$ will be the color of
$s$.  The line segment $q_{2i}$ will be cut from the long line segment $\ell$,
i.e., $q_{2i} = \frac{1}{N} - s$ and $q_{2i}$ will have the color of $\ell$.
Note that since $\ell \ge \frac{1}{N}$ there will be left-over line segment
(possibly of size zero) from $\ell$. The length of left-over line segment is
$\ell' = \ell - q_{2i}$. The left-over length $\ell'$ could be long (i.e. $1/N$
or longer) or short (less than $1/N$). Based on its length, we insert the
left-over line segment into the corresponding list.

First, notice that by construction $q_{2i-1}$ and $q_{2i}$ satisfy
\eqref{equation:two-segments}.  Second, notice that in each round we decrease
the number of line segments in the two piles by one: We remove $2$ line
segments and we insert one left-over line segment back. Thus, the algorithm in
exactly $N$ rounds removes all line segments from the two piles. Finally,
notice that at the beginning of round $i$ there are $N-i+1$ line segments in
the two piles and their length is $1 - \frac{i-1}{N}$, since in each of the
previous $i-1$ rounds we have decreased the total length by $1/N$. Therefore,
at the beginning of the round $i$, at least one line segment in the two piles
will have length at least
$$
\frac{1 - \frac{i-1}{N}}{N - i + 1}
= \frac{\frac{N - i + 1}{N}}{N - i + 1}
= \frac{1}{N} \; .
$$
In other words, there is at least one long line segment at the beginning of
every round.

It could happen that at the beginning of round $i$ there is no short line
segment.  That means all (long) line segments have length $1/N$. This follows
from that there are $N-i+1$ long line segments and their total length is
$\frac{N-i+1}{N}$. In that case, we can simply create $N-i+1$ pairs of line
segments $(q_{2i-1}, q_{2i}), (q_{2i+1}, q_{2i+2}), \dots, (q_{2N-1}, q_{2N})$.
The first line segment in each pair is constructed from a long line segment and
the second line segment in each pair has zero length of arbitrary color.

The two piles can be implemented using two stacks stored in the same array of
size $N$.  The stacks have their bottoms at the opposite ends of the array and
grow inward.

\bibliographystyle{plain}
\bibliography{biblio}

\end{document}
